
\فصل{قالب پیکربندی استخراج}
\label{appendix:config_format}

در این پیوست، ساختار فایل پیکربندی که برای تعریف فیلدهای هدف جهت استخراج اطلاعات به کار می‌رود، تشریح شده است.

این محتوا به عامل‌ها کمک می‌کند تا ساختار دقیق خروجی را درک کرده و اعتبارسنجی‌های لازم را انجام دهند. نمونه‌ای از ساختار این فایل در ادامه آمده است:

\begin{latin}
\begin{verbatim}
{
  "function_name": "extract_single_project",
  "object_name": "project",
  "fields": {
    "name": {"type": "string", "required": true},
    "company": {"type": "string", "required": true},
    "profit": {"type": "number", "required": true},
    "guarantee": {"type": "string", "required": true},
    "investment_amount": {"type": "string"},
    "duration": {"type": "string"},
    "status": {"type": "string"}
  }
}
\end{verbatim}
\end{latin}

\noindent
در ادامه، توضیح هر یک از این فیلدهای اطلاعاتی آمده است:
\begin{itemize}
    \item \lr{name}: عنوان رسمی طرح تأمین مالی.
    \item \lr{company}: نام شرکت، نهاد یا شخصی که متقاضی جذب سرمایه است.
    \item \lr{profit}: نرخ سود پیش‌بینی‌شده برای سرمایه‌گذاران که معمولاً به صورت درصد بیان می‌شود.
    \item \lr{guarantee}: نوع و جزئیات تضمین بازپرداخت اصل و سود سرمایه (مانند ضمانت‌نامه بانکی، بیمه و ...).
    \item \lr{investment amount}: مبلغ کل سرمایه مورد نیاز طرح یا حداقل مبلغ قابل سرمایه‌گذاری.
    \item \lr{duration}: مدت زمان اجرای طرح یا دوره بازپرداخت سرمایه.
    \item \lr{status}: وضعیت فعلی طرح (مانند در حال جذب سرمایه، تکمیل‌شده، فعال).
\end{itemize}


