
\فصل{نتیجه‌گیری و کارهای آینده}

این پژوهش یک معماری عامل‌محور جامع برای استخراج اطلاعات ساختارمند از صفحات وب تأمین مالی جمعی ارائه کرد. تمرکز اصلی بر طراحی و مقایسه سه راهبرد متفاوت استخراج اطلاعات بود که در قالب عامل‌های مستقل پیاده‌سازی شدند. نتایج به وضوح نشان داد که راهبردهای پیشرفته‌تر، مانند پیش‌پردازش هوشمند محتوا و ادغام نتایج چندمرحله‌ای که در عامل متخصص به کار گرفته شد، تأثیر چشمگیری بر افزایش دقت دارند، در حالی که روش‌های ساده‌تر مزایای خود را در سرعت و سهولت پیاده‌سازی حفظ می‌کنند.

\قسمت{جمع‌بندی دستاوردهای اصلی}

\begin{itemize}
    \item \textbf{دستیابی به دقت بالا از طریق عامل متخصص}: دستاورد اصلی این پژوهش، طراحی و پیاده‌سازی موفق عامل متخصص است که با دستیابی به دقت \textbf{\%$90.9$}، کارایی خود را در استخراج اطلاعات پیچیده به اثبات رساند. این نتیجه، ارزش ترکیب راهبردهای پیشرفته مانند پاک‌سازی هوشمند \lr{HTML}، استخراج چندمرحله‌ای و ادغام هوشمند نتایج را تأیید می‌کند.

    \item \textbf{تأثیر حیاتی پیش‌پردازش در بهینه‌سازی}: نشان داده شد که مرحله پاک‌سازی \lr{HTML} در عامل متخصص، با کاهش بیش از \textbf{\%$73$} در تعداد توکن‌های ورودی، نه تنها هزینه محاسباتی را به شدت کاهش می‌دهد، بلکه با حذف نویز، به عنوان یک عامل کلیدی در افزایش دقت نهایی نیز عمل می‌کند.

    \item \textbf{ارائه یک چارچوب تحلیلی برای مقایسه راهبردها}: این پژوهش با مقایسه عامل متخصص با دو عامل پایه، یک چارچوب کامل برای تحلیل موازنه بین دقت، سرعت و هزینه توکنی ارائه می‌دهد. این چارچوب به تصمیم‌گیری آگاهانه برای انتخاب راهبرد مناسب بر اساس نیازهای خاص هر کاربرد کمک می‌کند.

    \item \textbf{اعتبارسنجی روش‌شناسی ارزیابی مقیاس‌پذیر}: به عنوان یک دستاورد روش‌شناختی، این پژوهش نشان داد که می‌توان از یک داور خودکار مبتنی بر مدل زبانی بزرگ برای ارزیابی کارآمد و در مقیاس بزرگ استفاده کرد. داور خودکار با دستیابی به امتیاز \lr{F1} بالای ۹۰٪ در مقایسه با داوری انسانی، قابلیت اطمینان خود را به عنوان ابزاری برای تسریع چرخه‌های آزمایش و ارزیابی در پژوهش‌های آتی به اثبات رساند.
\end{itemize}

\قسمت{محدودیت‌های پژوهش}

علی‌رغم نتایج مثبت، این پژوهش با محدودیت‌هایی نیز همراه بود که مسیر را برای تحقیقات آتی هموار می‌کند:
\begin{itemize}
    \item \textbf{تعمیم‌پذیری به دامنه‌های دیگر}: اگرچه معماری ارائه‌شده انعطاف‌پذیر است، اما عملکرد آن به طور خاص بر روی دامنه تأمین مالی جمعی با زبان فارسی سنجیده شده است. ارزیابی عملکرد آن بر روی دامنه‌های دیگر با ساختارها و زبان‌های متفاوت نیازمند تحقیقات بیشتر است.
    \item \textbf{وابستگی به طراحی پرامپت}: کیفیت نتایج در عامل متخصص تا حد زیادی به کیفیت طراحی پرامپت‌ها وابسته است. مهندسی پرامپت یک فرآیند تکراری است و ممکن است پرامپت‌های بهینه‌تری نیز وجود داشته باشند.
\end{itemize}

\قسمت{پیشنهادها برای کارهای آینده}

بر اساس یافته‌ها و محدودیت‌های این پژوهش، مسیرهای متعددی برای تحقیقات آینده قابل تصور است:
\begin{itemize}
    \item \textbf{توسعه عامل‌های ترکیبی و تطبیق‌پذیر}: طراحی عاملی که بتواند به صورت خودکار و بر اساس پیچیدگی صفحه \lr{HTML}، بین راهبردهای مختلف (مثلاً سوئیچ از حالت پایه به متخصص) انتخاب کند، می‌تواند موازنه بهینه‌ای بین سرعت و دقت ایجاد کند.

    \item \textbf{بهبود الگوریتم ادغام هوشمند}: می‌توان الگوریتم ادغام در عامل متخصص را با در نظر گرفتن معیارهای پیچیده‌تر، مانند تحلیل معنایی مقادیر استخراج‌شده یا استفاده از مدل‌های زبانی برای قضاوت بین گزینه‌های مختلف، بهبود بخشید.

    \item \textbf{تحلیل عمیق‌تر خطاهای داور خودکار}: بررسی مواردی که داور خودکار با داور انسانی اختلاف نظر دارد (به ویژه موارد مثبت و منفی کاذب) می‌تواند به شناسایی نقاط ضعف مدل داور و بهبود دستورالعمل‌های ارزیابی آن منجر شود.

    \item \textbf{استفاده از مدل‌های چندوجهی}: بررسی استفاده از مدل‌هایی که علاوه بر متن، قادر به درک ساختار بصری صفحه وب نیز هستند، می‌تواند به استخراج اطلاعاتی که صرفاً از طریق تحلیل متنی قابل دستیابی نیستند، کمک کند.

\end{itemize}

در نهایت، این پژوهش نشان داد که با ترکیب هوشمندانه تکنیک‌های پردازش زبان طبیعی، معماری نرم‌افزار و روش‌شناسی ارزیابی دقیق، می‌توان به راه‌حل‌های کارآمد و قابل اعتمادی برای چالش‌های پیچیده استخراج اطلاعات از وب دست یافت.
