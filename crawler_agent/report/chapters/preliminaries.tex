
\فصل{مفاهیم اولیه}

در این فصل مفاهیم و مؤلفه‌های مورد استفاده در سامانه عامل‌محور استخراج اطلاعات معرفی می‌شوند.

\قسمت{تعاریف پایه}

\begin{itemize}
\item \textbf{طرح تأمین مالی جمعی}: روشی نوین برای جمع‌آوری سرمایه از تعداد زیادی از افراد است که معمولاً از طریق سکو‌های آنلاین انجام می‌شود. در این مدل، کارآفرینان، مخترعان یا تیم‌های پروژه‌ای، ایده‌ها و طرح‌های خود را در وب‌سایت‌های تخصصی منتشر می‌کنند و از عموم مردم درخواست سرمایه‌گذاری یا حمایت مالی می‌کنند. هر سرمایه‌گذار می‌تواند با مبالغ کوچک در پروژه سهیم شود و در ازای آن معمولاً مزایا یا سود مشخصی دریافت کند. این روش به افراد و کسب‌وکارهای نوپا اجازه می‌دهد بدون وابستگی به بانک‌ها یا سرمایه‌گذاران بزرگ، منابع مالی لازم برای توسعه محصول یا ایده خود را تأمین کنند و در عین حال بازخورد مستقیم بازار و جامعه را دریافت کنند. تأمین مالی جمعی علاوه بر تأمین سرمایه، ابزار مهمی برای اعتبارسنجی ایده‌ها، جذب مخاطب و ایجاد جامعه طرفداران اولیه پروژه نیز محسوب می‌شود.


\item \textbf{استخراج ساخت‌یافته}: تبدیل محتوای \lr{HTML} غیرساخت‌یافته به یک شیء داد‌ه‌ای با ساختار مشخص (\lr{JSON}).

\item \textbf{توکن}: واحد محاسباتی مصرف ورودی/خروجی در مدل‌های زبانی بزرگ که بر هزینه و زمان اثر مستقیم دارد.

\item \textbf{ماتریس اغتشاش}: یکی از ابزارهای کلیدی در ارزیابی عملکرد مدل‌های دسته‌بندی است که به‌صورت یک جدول دوبعدی نمایش داده می‌شود. در این ماتریس، سطرها نمایانگر کلاس‌های واقعی و ستون‌ها نمایانگر کلاس‌های پیش‌بینی‌شده توسط مدل هستند. هر خانهٔ ماتریس نشان‌دهنده تعداد نمونه‌هایی است که به‌طور خاص در آن ترکیب پیش‌بینی و واقعیت قرار گرفته‌اند. به کمک ماتریس اغتشاش می‌توان شاخص‌های مهمی مانند درستی پیش‌بینی (\lr{accuracy})، صحت (\lr{precision})، بازیابی (\lr{recall})  و  معیار\lr{F1} را محاسبه کرد و نقاط ضعف مدل در تشخیص هر کلاس را شناسایی نمود. این ابزار به‌ویژه در مسائل با ککلاس‌های نامتوازن اهمیت دارد، چرا که به جای تکیه صرف بر درصد کلی درست پیش‌بینی‌ها، دیدی دقیق نسبت به اشتباهات مدل و توزیع خطاها ارائه می‌دهد.
\end{itemize}

\قسمت{معماری کلی سامانه}

\زیرقسمت{عامل‌ها}

سامانه از سه عامل اصلی تشکیل شده است که هر کدام راهبردی متفاوت برای استخراج دارند:
\begin{itemize}
\item \textbf{عامل پایه (\lr{Basic})}: استخراج مستقیم با پرامپت ساده بدون به‌کارگیری فراخوانی تابع؛ تمرکز بر سادگی و سرعت.
\item \textbf{عامل تابع‌محور (\lr{Function})}: استفاده از فراخوانی تابع (\lr{Function Calling}) برای نگاشت مستقیم خروجی مدل به ساختار هدف؛ تضمین هم‌شکلی خروجی.
\item \textbf{عامل متخصص (\lr{Expert})}: شامل پاک‌سازی کارآمد \lr{HTML}، چند دور استخراج هوشمند، ارزیابی کیفیت فیلدها و ادغام هوشمند نتایج؛ تمرکز بر دقت بالاتر.
\end{itemize}

\زیرقسمت{سامانه‌های کمکی}

برای پشتیبانی از چرخه ارزیابی و مقایسه، چند جزء کمکی پیاده‌سازی شده است:
\begin{itemize}
\item \textbf{مقایسه‌ی عامل‌ها}: اجرای سه عامل روی هر پروژه و ثبت زمان پردازش و خروجی‌ها در فایل‌های مقایسه؛ امکان مشاهده سریع‌ترین عامل و موفقیت/شکست.
\item \textbf{راستی‌آزمایی/اعتبارسنجی}: دو مسیر موازی در نظر گرفته شده است؛ (الف) داوری انسانی با ثبت صحت هر فیلد در سطح پروژه، (ب) داور خودکار مبتنی بر مدل زبانی بزرگ برای سنجش خودکار همان فیلدها.
\item \textbf{محاسبه‌ی امتیازها و گزارش‌ها}: تولید گزارش دقت و آمار کل برای هر عامل (دقت، تعداد صحیح/نادرست، زمان میانگین)، گزارش کاهش توکن (خام در برابر پس از پاک‌سازی)، و گزارش ماتریس اغتشاش برای مقایسه‌ی داور خودکار با انسانی.
\end{itemize}

\قسمت{تکنولوژی‌ها}

\زیرقسمت{زبان برنامه‌نویسی}

هسته سامانه با \textbf{\lr{Python}} پیاده‌سازی شده است؛ به‌دلیل اکوسیستم بالغ برای پردازش داده و متن، سهولت یکپارچه‌سازی با رابط‌های مدل‌های زبانی، سرعت توسعه، قابل‌انتقال بودن بین سکوهای اجرا و پشتوانه کتابخانه‌های پایدار. در این پروژه از مجموعه‌ای از کتابخانه‌ها استفاده شده است:
\begin{itemize}
\item \textbf{\lr{google.generativeai}}: برقراری ارتباط با مدل‌های \textit{\lr{Gemini}}، ارسال پرامپت، دریافت پاسخ و به‌کارگیری \lr{Function Calling}.
\item \textbf{\lr{vertexai.preview.tokenization}}: محاسبه تقریبی تعداد توکن برای برآورد هزینه/زمان پیش از فراخوانی مدل.
\item \textbf{\lr{dotenv}}: مدیریت امن کلید دسترسی از طریق متغیرهای محیطی و تفکیک پیکربندی از کد.
\item \textbf{\lr{json}}: خواندن/نوشتن خروجی‌های ساختارمند عامل‌ها و گزارش‌های تجمیعی.
\item \textbf{\lr{os}} و \textbf{\lr{pathlib}}: مدیریت مسیرها و فایل‌ها (ورودی \lr{HTML}، خروجی نتایج، گزارش‌ها).
\item \textbf{\lr{time}} و \textbf{\lr{datetime}}: اندازه‌گیری زمان پردازش و درج زمان‌نگار (برچسب زمانی) در گزارش‌ها.
\item \textbf{\lr{re}}: پاک‌سازی و نرمال‌سازی \lr{HTML} با عبارت‌های باقاعده در عامل متخصص.
\item \textbf{\lr{typing}}: مشخص‌سازی نوع‌ها برای افزایش خوانایی و نگهداشت‌پذیری کد.
\end{itemize}

\زیرقسمت{مدل زبانی}

\زیرزیرقسمت{مدل \lr{Gemini}}

\textbf{\lr{Gemini}} خانواده‌ای از مدل‌های زبانی چندوجهیِ عمومی و مقیاس‌پذیر است که برای درک و تولید متن، استدلال روی داده‌های ساخت‌یافته و نیمه‌ساخت‌یافته، و کار با ورودی‌های متنی طولانی توسط گوگل طراحی شده است. این خانواده، علاوه بر پشتیبانی از متن، برای یکپارچه‌سازی با کارکردهای بیرونی (مانند فراخوانی تابع) و پیروی از قالب‌های از پیش تعیین‌شده توانمندی‌های مناسبی دارد؛ ویژگی‌ای که آن را برای \textit{استخراج ساختارمند} از محتوای \lr{HTML} مناسب می‌سازد. به طور خلاصه، \lr{Gemini} ترکیبی از سرعت (به‌ویژه در گونه‌های \lr{flash/flash-lite}) و کنترل‌پذیری ساختار از طریق فراخوانی تابع را فراهم می‌کند.

مزیت استفاده از \textbf{\lr{Gemini}} برای استخراج ساختارمند از \lr{HTML}:
\begin{itemize}
\item تاب‌آوری نسبت به نویزهای متنی و نشانه‌گذاری‌های زائد صفحات وب
\item توانایی پیروی از الگو و تولید خروجی با ساختار مشخص
\item پشتیبانی بومی از فراخوانی تابع برای کنترل‌پذیری بهتر خروجی
\end{itemize}

به‌طور کلی خانواده \textit{\lr{Gemini}} شامل گونه‌های \textit{\lr{pro}}، \textit{\lr{flash}} و \textit{\lr{flash-lite}} است. که در این پروژه   گونه‌ی \textit{\lr{flash-lite}} به‌دلایل زیر انتخاب شده‌است:
\begin{itemize}
\item کاهش هزینه و زمان پردازش هنگام اجرای انبوه بر روی ده‌ها صفحه \lr{HTML}
\item کفایت کیفیت برای استخراج فیلدهای ساختاری نسبتاً کوتاه
\item سازگاری مناسب با فراخوانی تابع و قالب‌های ساخت‌یافته
\end{itemize}

\زیرزیرقسمت{مدل \lr{Gemma-3-27b-it} برای داوری}

\textbf{\lr{Gemma}} خانواده‌ای دیگر از مدل‌های زبانی گوگل است که نسخه \lr{27B-it} آن (آموزش‌دیده برای پیروی از دستورالعمل) توازن مناسبی میان توان استدلال، پایبندی به دستورالعمل ارزیابی و زمان پاسخ ارائه می‌کند. در این پروژه برای \textbf{داوری خودکار} از \lr{gemma-3-27b-it} استفاده شده است (مطابق گزارش‌های داوری خودکار). دلایل انتخاب:
\begin{itemize}
\item \textbf{دقت و ثبات در پیروی از دستورالعمل ارزیابی}: مناسب برای سناریوهای \lr{درست/نادرست} در سطح فیلد.
\item \textbf{هزینه/زمان مناسب}: امکان ارزیابی انبوه پروژه‌ها با تأخیر و هزینه قابل‌قبول.
\item \textbf{سازگاری با روبریک و قالب}: توان تولید پاسخ یکنواخت برای محاسبه خودکار معیارها (مانند ماتریس اغتشاش).
\end{itemize}

\زیرقسمت{فراخوانی تابع (\lr{Function Calling})}

در این رویکرد، ساختار هدف به‌صورت تابعی با آرگومان‌های نام‌دار تعریف می‌شود؛ مدل به‌جای متن آزاد، آرگاران‌های تابع را مقداردهی می‌کند. مزایای این روش شامل موارد زیر است:
\begin{itemize}
\item \textbf{هم‌شکلی خروجی}: تمام پاسخ‌ها در قالب یک ساختار یکنواخت بازمی‌گردند.
\item \textbf{کاهش خطاهای قالب}: احتمال تولید متن آزاد نامعتبر یا کلیدهای نامتعارف کاهش می‌یابد.
\item \textbf{یکپارچگی با اعتبارسنجی}: کنترل نوع/وجود هر فیلد ساده‌تر شده و ارزیابی خودکار کیفیت میسر می‌شود.
\end{itemize}

\قسمت{داوری و ارزیابی}

\begin{itemize}
\item \textbf{داوری انسانی}: صحت فیلدهای استخراج‌شده توسط هر عامل و برای هر طرح تامین مالی جمعی ثبت می‌شود.
\item \textbf{داور خودکار}: ارزیابی همان فیلدها توسط یک مدل زبانی بزرگ (\lr{LLM}) بر اساس معیارِ درست/نادرست و مطابق دستورالعمل ارزیابی. دلایل استفاده از داوری خودکار:
\begin{itemize}
\item \textbf{مقیاس‌پذیری و سرعت}: ارزیابی تعداد زیاد خروجی‌‌ها در زمان کم.
\item \textbf{یکنواختی}: کاهش تنوع قضاوت انسانی و افزایش ثبات نتایج.
\item \textbf{کاهش هزینه}: جایگزینی بخشی از نیروی انسانی در ارزیابی‌های تکراری.
\item \textbf{پایش پیوسته}: امکان ارزیابی مداوم با به‌روزرسانی سریع گزارش‌ها.
\end{itemize}
\textit{کاربرد ویژه در این پروژه}: به‌دلیل تعداد پروژه‌ها، تغییر مستمر صفحات و نیاز به بازخورد سریع، داوری خودکار زمان تیم را به‌طور معنادار کاهش داده و مسیر آزمایش/مقایسه عامل‌ها را تسریع می‌کند.
\end{itemize}

