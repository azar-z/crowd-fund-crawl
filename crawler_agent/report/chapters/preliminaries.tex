
\فصل{مفاهیم اولیه و تعاریف}

در این فصل، مفاهیم بنیادی و چارچوب‌های نظری که اساس معماری و روش‌شناسی این پژوهش را تشکیل می‌دهند، معرفی می‌شوند. تمرکز بر تعریف مفاهیمی است که برای درک عمیق راهکار ارائه‌شده ضروری هستند، پیش از آنکه جزئیات پیاده‌سازی در فصل‌های بعدی مورد بحث قرار گیرند.

\قسمت{طرح تأمین مالی جمعی}
تأمین مالی جمعی روشی نوین برای جذب سرمایه است که در آن، کارآفرینان و صاحبان ایده، طرح‌های خود را از طریق سکوهای آنلاین به تعداد زیادی از سرمایه‌گذاران بالقوه عرضه می‌کنند. هر فرد می‌تواند با سرمایه‌گذاری مبالغ کوچک در این طرح‌ها مشارکت کرده و در ازای آن، متناسب با مدل طرح، سود، سهام یا پاداش دریافت کند. این سازوکار، دسترسی به منابع مالی را برای کسب‌وکارهای نوپا تسهیل کرده و علاوه بر تأمین سرمایه، به عنوان ابزاری برای اعتبارسنجی بازار و ایجاد جامعه اولیه حامیان عمل می‌کند. استخراج دقیق اطلاعات از صفحات این طرح‌ها برای تحلیل بازار و تصمیم‌گیری سرمایه‌گذاران از اهمیت بالایی برخوردار است.

\قسمت{استخراج اطلاعات ساختارمند}

استخراج اطلاعات شاخه‌ای از پردازش زبان طبیعی است که هدف آن استخراج خودکار اطلاعات ساختارمند از متون غیرساختارمند است. در چارچوب این پروژه، "اطلاعات ساختارمند" به داده‌هایی اطلاق می‌شود که در یک ساختار از پیش‌تعریف‌شده (مانند یک \lr{JSON} با کلیدهای مشخص) قرار می‌گیرند و "متون غیرساختارمند" همان محتوای خام صفحات \lr{HTML} است. هدف نهایی، تبدیل یک سند \lr{HTML} پیچیده و پُرنویز به یک رکورد داده تمیز و قابل استفاده در پایگاه داده است.

\قسمت{مدل‌های زبانی بزرگ (\lr{LLMs})}

مدل‌های زبانی بزرگ، سیستم‌های هوش مصنوعی مبتنی بر شبکه‌های عصبی عمیق هستند که بر روی حجم عظیمی از داده‌های متنی آموزش دیده‌اند. ویژگی کلیدی آن‌ها، توانایی درک، تولید، خلاصه‌سازی و استدلال بر روی زبان طبیعی است.

\زیرقسمت{مدل \lr{Gemini} برای استخراج اطلاعات}
خانواده مدل‌های \lr{Gemini} که توسط گوگل توسعه داده شده، به طور خاص برای کاربردهای چندوجهی و درک متون طولانی طراحی شده است. مدل‌های این خانواده، به ویژه گونه‌های سبک‌تر مانند \lr{Gemini Flash}، توازن بسیار خوبی بین هزینه، سرعت و دقت برقرار می‌کنند. قابلیت کلیدی آن‌ها برای این پروژه، پشتیبانی قوی از \textbf{فراخوانی تابع} است که به مدل اجازه می‌دهد خروجی خود را مستقیماً در یک ساختار از پیش‌تعریف‌شده و معتبر تولید کند. این ویژگی، \lr{Gemini} را به ابزاری ایده‌آل برای وظیفه اصلی این پژوهش، یعنی استخراج اطلاعات ساختارمند، تبدیل می‌کند.

\زیرقسمت{مدل \lr{Gemma} برای ارزیابی}
خانواده مدل‌های \lr{Gemma} نیز از مدل‌های متن‌باز و قدرتمند گوگل هستند. در این پروژه از مدل \lr{Gemma-3-27b-it}، که یک نسخه آموزش‌دیده برای پیروی از دستورالعمل است، برای وظیفه \textbf{داوری خودکار} استفاده شده است. انتخاب یک مدل نسبتاً سبک‌تر برای این وظیفه عمدی بوده است؛ چرا که وظایف اعتبارسنجی و قضاوت دودویی (صحیح/غلط) معمولاً به اندازه استخراج اولیه نیازمند توان محاسباتی بالا نیستند و استفاده از مدل‌های بهینه‌تر می‌تواند هزینه ارزیابی‌های مقیاس‌بزرگ را به شدت کاهش دهد.

\قسمت{فراخوانی تابع}

فراخوانی تابع یک قابلیت پیشرفته در مدل‌های زبانی بزرگ است که به آن‌ها اجازه می‌دهد خروجی خود را به جای متن آزاد، در قالب یک فراخوانی تابع ساختارمند تولید کنند. در این روش، یک ساختار تابع (شامل نام، توضیحات و پارامترها با نوع مشخص) به مدل ارائه می‌شود. مدل پس از پردازش ورودی، مؤلفه‎‌های این تابع را با اطلاعات استخراج‌شده پر کرده و یک شیء ساختارمند بازمی‌گرداند. این مکانیزم، پایایی و قابلیت اطمینان خروجی‌های ساختارمند را به شدت افزایش می‌دهد.

\قسمت{معماری عامل‌محور}

یک سیستم عامل‌محور، سیستمی است که از مجموعه‌ای از عامل‌های مستقل و هوشمند تشکیل شده است. هر عامل قادر است محیط خود را درک کرده و برای رسیدن به اهداف مشخصی، به صورت مستقل عمل کند. در این پژوهش، از این معماری برای پیاده‌سازی و مقایسه راهبردهای مختلف استخراج استفاده شده است. هر "عامل" یک رویکرد خاص برای استخراج اطلاعات را کپسوله می‌کند. این طراحی امکان تحلیل مستقل هر رویکرد را فراهم می‌سازد:
\begin{itemize}
    \item \textbf{عامل پایه}: نماینده رویکرد سریع و ساده مبتنی بر پرامپت.
    \item \textbf{عامل تابع‌محور}: نماینده رویکردی که بر پایایی و یکنواختی ساختاری خروجی تمرکز دارد و از قابلیت فراخوانی تابع مدل زبانی استفاده می‌کند.
    \item \textbf{عامل متخصص}: نماینده رویکردی پیچیده‌تر که با پیش‌پردازش و پس‌پردازش، به دنبال حداکثرسازی دقت است.
\end{itemize}
مقایسه این سه عامل، امکان درک عمیق موازنه بین دقت، سرعت و پیچیدگی پیاده‌سازی را فراهم می‌آورد.

\قسمت{مدل زبانی بزرگ به عنوان داور}
ارزیابی عملکرد سیستم‌های استخراج اطلاعات به طور سنتی به داوری انسانی وابسته است که فرآیندی کند، پرهزینه و مقیاس‌ناپذیر است. الگوواره \textbf{«مدل زبانی بزرگ به عنوان داور»} یک راهکار نوین برای این چالش است. در این رویکرد، از یک مدل زبانی توانمند (در این پروژه \lr{Gemma}) به عنوان یک داور خودکار برای ارزیابی خروجی‌های تولیدشده توسط عامل‌های دیگر استفاده می‌شود. این داور با دریافت متن منبع، خروجی استخراج‌شده و یک معیار ارزیابی دقیق، قضاوت می‌کند که آیا خروجی صحیح است یا خیر. مزایای اصلی این روش شامل مقیاس‌پذیری بالا، سرعت و یکنواختی در ارزیابی است که آن را به ابزاری کارآمد برای پایش مداوم و آزمایش‌های گسترده تبدیل می‌کند.

\قسمت{ماتریس اغتشاش و معیارهای ارزیابی}

ماتریس اغتشاش ابزاری استاندارد برای ارزیابی عملکرد سیستم‌های طبقه‌بندی است. در یک مسئله طبقه‌بندی دودویی (مانند تشخیص "صحیح" یا "نادرست" بودن یک فیلد استخراج‌شده)، این ماتریس چهار مقدار کلیدی را نمایش می‌دهد:
\begin{itemize}
	\item \textbf{مثبت واقعی}: تعداد مواردی که به درستی "صحیح" پیش‌بینی شده‌اند.
	\item \textbf{مثبت کاذب}: تعداد مواردی که به اشتباه "صحیح" پیش‌بینی شده‌اند (در حالی که نادرست بوده‌اند).
	\item \textbf{منفی واقعی}: تعداد مواردی که به درستی "نادرست" پیش‌بینی شده‌اند.
	\item \textbf{منفی کاذب}: تعداد مواردی که به اشتباه "نادرست" پیش‌بینی شده‌اند (در حالی که صحیح بوده‌اند).
\end{itemize}
بر اساس این مقادیر، معیارهای عملکردی مهمی مانند دقت (\lr{Accuracy})، صحت (\lr{Precision})، بازیابی (\lr{Recall}) و معیار \lr{F1} محاسبه می‌شوند که تصویر جامعی از عملکرد سیستم ارزیابی ارائه می‌دهند.
