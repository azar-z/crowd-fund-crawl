
\فصل{مقدمه}

این گزارش به تشریح طراحی و پیاده‌سازی یک سامانه عامل‌محور برای استخراج خودکار اطلاعات طرح‌های «تأمین مالی جمعی» از صفحات وب می‌پردازد.


\قسمت{تعریف مسئله}

\زیرقسمت{بیان مسئله در سطح کلان}

استخراج اطلاعات ساخت‌یافته از صفحات وب غیرساخت‌یافته (HTML) یکی از مسائل کلاسیک و در عین حال چالش‌برانگیز است. تعدد قالب‌ها، وجود نویزهای فراوان (اسکریپت‌ها، استایل‌ها، تبلیغات و ...)، ناهمگنی نشانه‌گذاری‌ها و تفاوت زبان و نگارش محتوا باعث می‌شود تبدیل \lr{HTML} به داده‌های تمیز و ساختارمند نیازمند رویکردهای هوشمندانه باشد.

\زیرقسمت{بیان مسئله در دامنه خاص}

در این پروژه دامنه هدف، سکو‌های ایرانی طرح‌های «تأمین مالی جمعی» است. هدف، استخراج خودکار اطلاعات کلیدی هر طرح شامل نام طرح، شرکت/ناشر، سود مورد انتظار، مدت، نوع تضمین، وضعیت و (در صورت وجود) مبلغ سرمایه‌گذاری از محتوای \lr{HTML} خام صفحه هر طرح است؛ به‌نحوی که خروجی نهایی ساخت‌یافته و قابل تحلیل باشد.

\قسمت{اهمیت موضوع}

\begin{itemize}
\item  \textbf{کارایی و مقیاس‌پذیری}: جایگزینی فرایند دستی پرهزینه با پردازش خودکار، امکان پوشش سکو‌های متعدد و طرح‌های بسیار را در زمان کمتری فراهم می‌کند.

\item  \textbf{پایایی و یکنواختی}: تبدیل داده‌های ناهمگن به ساختار واحد، تحلیل‌پذیری و مقایسه‌پذیری را افزایش می‌دهد.

\item  \textbf{اهمیت صحت اطلاعات برای تحلیل و انتشار}: دقت استخراج برای تحلیل‌های آتی، گزارش‌دهی عمومی، تصمیم‌سازی سرمایه‌گذاران و حتی انطباق‌های نظارتی حیاتی است؛ خطا در داده‌های پایه می‌تواند به برداشت‌ها و تصمیم‌های اشتباه منجر شود.

\item \textbf{پایش پیوسته و به موقع اطلاعات}: در پایش دستی اطلاعات، از زمان انتشار اطلاعات تا دسترسی سامانه به اطلاعات مورد نیاز تاخیر زمانی زیادی وجود دارد. در دامنه‌ای مانند دامنه‌ی سرمایه‌گذاری و تامین مالی جمعی، دسترسی سریع و به موقع به طرح‌هایی که تازه منتشره شده‌اند اهمیت پیدا می‌کند. استخراج خودکار اطلاعات می‌تواند زمان دسترسی به اطلاعات را به شدت کاهش دهد.
\end{itemize}

\قسمت{چالش‌ها}

\begin{itemize}
\item ناهمگونی شدید قالب و نشانه‌گذاری صفحات وب میان سکو‌های مختلف
\item زبان فارسی و تغییرات نگارشی/زبانی در متن‌ها
\item نویزهای ساختاری (\lr{script/style/ads}) و محتوایی در \lr{HTML}
\item نیاز به توازن بین دقت استخراج، زمان پردازش و هزینه توکنی
\end{itemize}


\قسمت{اهداف و دستاوردها}

\begin{itemize}
\item طراحی و پیاده‌سازی سه عامل استخراج: \textbf{عامل پایه} (\lr{Basic})، \textbf{عامل تابع‌محور} (\lr{Function}) و \textbf{عامل متخصص} (\lr{Expert})
\item پیاده‌سازی گام پاک‌سازی کارآمد \lr{HTML} برای کاهش چشم‌گیر توکن‌های ورودی
\item ارزیابی خروجی عامل‌ها با \textbf{داوری انسانی} و \textbf{داور \lr{LLM}} و گزارش جامع مقایسه‌ای
\item ارائه تحلیل کارایی (دقت، زمان، هزینهٔ توکنی)
\item سنجش توافق داور \lr{LLM} با انسان از طریق \textbf{ماتریس اغتشاش}
\end{itemize}