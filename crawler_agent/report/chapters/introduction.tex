
\فصل{مقدمه}

این پژوهش به طراحی، پیاده‌سازی و ارزیابی یک سامانه هوشمند عامل‌محور برای استخراج خودکار و ساختارمند اطلاعات از صفحات وب «تأمین مالی جمعی» می‌پردازد. با توجه به رشد روزافزون سکوهای تأمین مالی جمعی و نیاز مبرم به تحلیل داده‌های مرتبط با طرح‌های سرمایه‌گذاری، فرآیند دستی استخراج اطلاعات به دلیل زمان‌بر بودن، هزینه بالا و مستعد خطا بودن، دیگر پاسخگوی نیازهای موجود نیست. این گزارش یک راهکار مبتنی بر مدل‌های زبانی بزرگ را معرفی می‌کند که با بهره‌گیری از معماری عامل‌محور، رویکردهای متفاوتی برای استخراج اطلاعات را پیاده‌سازی و مقایسه می‌نماید.


\قسمت{تعریف مسئله}

\زیرقسمت{بیان مسئله در سطح کلان}

استخراج اطلاعات ساخت‌یافته از صفحات وب غیرساخت‌یافته (\lr{HTML}) یکی از مسائل کلاسیک و در عین حال چالش‌برانگیز است. تعدد قالب‌ها، وجود نویزهای فراوان (اسکریپت‌ها، استایل‌ها، تبلیغات و ...)، ناهمگنی نشانه‌گذاری‌ها و تفاوت زبان و نگارش محتوا باعث می‌شود تبدیل \lr{HTML} به داده‌های تمیز و ساختارمند نیازمند رویکردهای هوشمندانه باشد.

\زیرقسمت{بیان مسئله در دامنه خاص}

در این پروژه دامنه هدف، سکو‌های ایرانی طرح‌های «تأمین مالی جمعی» است. هدف، استخراج خودکار اطلاعات کلیدی هر طرح شامل نام طرح، شرکت متقاضی، سود مورد انتظار، مدت، نوع تضمین، وضعیت و مبلغ سرمایه‌گذاری از محتوای \lr{HTML} خام صفحه هر طرح است؛ به‌نحوی که خروجی نهایی ساخت‌یافته و قابل تحلیل باشد.

\قسمت{اهمیت موضوع}

\begin{itemize}
\item  \textbf{کارایی و مقیاس‌پذیری}: جایگزینی فرایند دستی پرهزینه با پردازش خودکار، امکان پوشش سکو‌های متعدد و طرح‌های بسیار را در زمان کمتری فراهم می‌کند.

\item  \textbf{پایایی و یکنواختی}: تبدیل داده‌های ناهمگن به ساختار واحد، تحلیل‌پذیری و مقایسه‌پذیری را افزایش می‌دهد.

\item  \textbf{اهمیت صحت اطلاعات برای تحلیل و انتشار}: دقت استخراج برای تحلیل‌های آتی، گزارش‌دهی عمومی، تصمیم‌سازی سرمایه‌گذاران و حتی انطباق‌های نظارتی حیاتی است؛ خطا در داده‌های پایه می‌تواند به برداشت‌ها و تصمیم‌های اشتباه منجر شود.

\item \textbf{پایش پیوسته و به موقع اطلاعات}: در پایش دستی اطلاعات، از زمان انتشار اطلاعات تا دسترسی سامانه به اطلاعات مورد نیاز تاخیر زمانی زیادی وجود دارد. در دامنه‌ای مانند دامنه‌ی سرمایه‌گذاری و تامین مالی جمعی، دسترسی سریع و به موقع به طرح‌هایی که تازه منتشره شده‌اند اهمیت پیدا می‌کند. استخراج خودکار اطلاعات می‌تواند زمان دسترسی به اطلاعات را به شدت کاهش دهد.
\end{itemize}

\قسمت{چالش‌ها}

\begin{itemize}
\item ناهمگونی شدید قالب و نشانه‌گذاری صفحات وب میان سکو‌های مختلف
\item زبان فارسی و تغییرات نگارشی/زبانی در متن‌ها
\item نویزهای ساختاری (\lr{script}، \lr{style}، \lr{ads}) و محتوایی در \lr{HTML}
\item نیاز به توازن بین دقت استخراج، زمان پردازش و هزینه توکنی
\end{itemize}


\قسمت{اهداف پژوهش}
هدف اصلی این پژوهش، طراحی و ساخت یک \textbf{عامل متخصص} با دقت بالا برای استخراج اطلاعات ساختارمند از صفحات وب تأمین مالی جمعی بود. این هدف از طریق ترکیب سه روش کلیدی دنبال شد: (۱) پیش‌پردازش و پاک‌سازی هوشمند محتوای \lr{HTML} برای کاهش نویز و هزینه، (۲) استخراج اطلاعات در چندین دور با پرامپت‌های متنوع برای افزایش پوشش و دقت، و (۳) پس‌پردازش و ادغام هوشمند نتایج برای دستیابی به یک خروجی نهایی بهینه.

\noindent
برای ارزیابی میزان اثربخشی عامل متخصص، عملکرد آن با دو راهبرد ساده‌تر (عامل پایه و عامل تابع‌محور) که به عنوان خط پایه عمل می‌کردند، مقایسه شد. این تحلیل مقایسه‌ای، موازنه بین دقت، سرعت و پیچیدگی پیاده‌سازی را به وضوح نشان می‌دهد. به عنوان یک هدف پژوهشی ثانویه، کارایی الگوواره «مدل زبانی بزرگ به عنوان داور» نیز برای اعتبارسنجی خودکار نتایج مورد بررسی قرار گرفت تا پتانسیل آن برای تسریع چرخه‌های ارزیابی در تحقیقات آینده سنجیده شود.