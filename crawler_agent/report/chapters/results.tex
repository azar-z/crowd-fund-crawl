
\فصل{نتایج تجربی}

در این فصل، نتایج به‌دست‌آمده از اجرای سه عامل بر روی مجموعه داده ارزیابی و تحلیل می‌شود. ابتدا عملکرد عامل‌ها در استخراج اطلاعات مقایسه شده و سپس، به عنوان یک تحلیل روش‌شناختی، عملکرد داور خودکار در مقایسه با داوری انسانی سنجیده می‌شود.

\قسمت{تحلیل کارایی پیش‌پردازش: کاهش توکن}

یکی از فرضیه‌های اصلی در طراحی عامل متخصص، تأثیر مثبت پاک‌سازی \lr{HTML} بر کاهش هزینه و افزایش تمرکز مدل بود. جدول \ref{tab:token_reduction} تأثیر این گام را بر تعداد توکن‌های ورودی نشان می‌دهد.

\begin{table}[h!]
\centering
\caption{مقایسه حجم توکن ورودی قبل و بعد از پاک‌سازی \lr{HTML}}
\label{tab:token_reduction}
\begin{tabular}{|l|c|}
\hline
\textbf{شاخص توکن} & \textbf{مقدار} \\
\hline
مجموع توکن خام (ورودی عامل پایه) & $2,921,762$ \\
مجموع توکن پاک‌شده (ورودی عامل متخصص) & $765,006$ \\
\hline
\textbf{کاهش کل توکن} & \textbf{$2,156,756$ (\%$73.82$)} \\
\hline
میانگین توکن به‌ازای هر پروژه (خام) & $56,188$ \\
میانگین توکن به‌ازای هر پروژه (پاک‌شده) & $14,712$ \\
\hline
\textbf{میانگین کاهش توکن ورودی به ازای هر پروژه} & \textbf{$41,476$} \\
\hline
\end{tabular}
\end{table}

\noindent
تحلیل نتایج نشان می‌دهد که گام پیش‌پردازش به طور میانگین حجم ورودی را \textbf{بیش از ۷۳ \%} کاهش داده است. این کاهش چشمگیر نه تنها هزینه فراخوانی مدل‌های زبانی را به شدت کاهش می‌دهد، بلکه با حذف اطلاعات نامرتبط، به مدل کمک می‌کند تا بر روی محتوای اصلی تمرکز کرده و دقت استخراج را بهبود بخشد، که این موضوع در بخش بعدی مشهود است.

\قسمت{مقایسه عملکرد عامل‌های استخراج}
برای ارزیابی و مقایسه، هر سه عامل بر روی مجموعه داده‌ای یکسان شامل ۵۲ نمونه طرح از سکوهای مختلف تأمین مالی جمعی اجرا شدند. جدول \ref{tab:agent_performance} عملکرد سه عامل را بر اساس معیارهای کلیدی دقت (بر اساس داوری انسانی) و میانگین زمان پردازش به ازای هر پروژه مقایسه می‌کند.

\begin{table}[h!]
\centering
\caption{مقایسه عملکرد سه عامل استخراج اطلاعات}
\label{tab:agent_performance}
\begin{tabular}{|l|c|c|}
\hline
\textbf{عامل} & \textbf{دقت خروجی (درصد)} & \textbf{میانگین زمان پردازش (ثانیه)} \\
\hline
عامل پایه & $78.0$ & $2.93$ \\
\hline
عامل تابع‌محور & $76.9$ & $2.06$ \\
\hline
عامل متخصص & \textbf{$90.9$} & $7.59$ \\
\hline
\end{tabular}
\end{table}

\noindent
از نتایج فوق می‌توان چند نکته کلیدی را استنتاج کرد:
\begin{itemize}
    \item \textbf{برتری عامل متخصص}: عامل متخصص با اختلاف قابل توجهی (بیش از ۱۲ درصد) دقیق‌ترین نتایج را تولید کرده است. این برتری مستقیماً به معماری چندمرحله‌ای آن، به ویژه گام پاک‌سازی \lr{HTML} و ادغام هوشمند نتایج، نسبت داده می‌شود.
    \item \textbf{موازنه سرعت و دقت}: عامل تابع‌محور و عامل پایه دقت تقریباً یکسانی داشتند، اما عامل تابع‌محور به طور قابل توجهی سریع‌تر بود. این ویژگی، عامل تابع‌محور را به گزینه‌ای مناسب برای کاربردهایی تبدیل می‌کند که در آن‌ها سرعت پاسخ‌دهی بر دقت مطلق اولویت دارد.
    \item \textbf{قدرت پیروی از دستورالعمل در مدل}: دقت بالای عامل پایه ($78\%$)، با وجود عدم استفاده از فراخوانی تابع، نشان‌دهنده قدرت بالای مدل \lr{Gemini} در پیروی از دستورالعمل‌ها و تولید خروجی با ساختار صحیح است. این نتیجه نشان می‌دهد که حتی بدون سازوکارهای سخت‌گیرانه فراخوانی تابع، مدل قادر است در اکثر موارد، خروجی \lr{JSON} معتبر و مطابق با الگوی درخواستی تولید کند.
    \item \textbf{هزینه دقت}: افزایش چشمگیر دقت در عامل متخصص با هزینه زمانی بالاتری همراه است. این زمان اضافی صرف مراحل پیش‌پردازش، چندین دور فراخوانی مدل و پس‌پردازش (ادغام) می‌شود.
\end{itemize}

\قسمت{ارزیابی عملکرد داور خودکار}
به عنوان هدف پژوهشی ثانویه، عملکرد داور خودکار در مقایسه با داوری انسانی سنجیده شد تا قابلیت اطمینان آن برای پژوهش‌های آتی مشخص شود. جدول \ref{tab:judge_comparison} ابتدا امتیازهای دقت اختصاص‌داده‌شده به هر عامل توسط هر دو داور را نمایش می‌دهد تا میزان هم‌خوانی کلی آن‌ها مشخص شود.

\begin{table}[h!]
\centering
\caption{مقایسه امتیاز دقت هر عامل از دیدگاه داور انسانی و داور خودکار}
\label{tab:judge_comparison}
\begin{tabular}{|l|c|c|}
\hline
\textbf{عامل} & \textbf{امتیاز داور انسانی (درصد)} & \textbf{امتیاز داور خودکار (درصد)} \\
\hline
عامل پایه & $78.0$ & $65.7$ \\
\hline
عامل تابع‌محور & $76.9$ & $69.0$ \\
\hline
عامل متخصص & \textbf{$90.9$} & \textbf{$82.7$} \\
\hline
\end{tabular}
\end{table}

\noindent
این مقایسه نشان می‌دهد که اگرچه داور خودکار در تشخیص عامل برتر (عامل متخصص) با داور انسانی هم‌نظر است، اما به طور کلی تمایل دارد امتیازهای پایین‌تری را ثبت کند. 

\noindent
برای تحلیل دقیق‌تر عملکرد داور خودکار، جدول \ref{tab:judge_performance} ماتریس اغتشاش و معیارهای کلیدی حاصل از مقایسه قضاوت آن با داور انسانی را نشان می‌دهد.

\begin{table}[h!]
\centering
\caption{ماتریس اغتشاش و معیارهای ارزیابی داور خودکار}
\label{tab:judge_performance}
\begin{tabular}{|l|c|}
\hline
\textbf{معیار} & \textbf{مقدار} \\
\hline
\textbf{دقت (\lr{Accuracy})} & \textbf{$0.854$} \\
\textbf{دقت (\lr{Precision})} & $0.965$ \\
\textbf{بازخوانی (\lr{Recall})} & $0.853$ \\
\textbf{معیار \lr{F1}} & \textbf{$0.905$} \\
\hline
\hline
\textbf{مثبت‌های درست (\lr{TP})} & $763$ \\
\textbf{منفی‌های درست (\lr{TN})} & $169$ \\
\textbf{مثبت‌های کاذب (\lr{FP})} & $28$ \\
\textbf{منفی‌های کاذب (\lr{FN})} & $132$ \\
\hline
\end{tabular}
\end{table}

\noindent
نتایج ماتریس اغتشاش نشان می‌دهد که داور خودکار با امتیاز \lr{F1} بالای ۹۰٪، توافق بالایی با داوری انسانی دارد. این سطح از قابلیت اطمینان، استفاده از آن را به عنوان یک ابزار کارآمد برای ارزیابی سریع و در مقیاس بزرگ در پژوهش‌های آتی توجیه می‌کند.
